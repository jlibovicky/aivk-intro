\begin{frame}{Organizace předmětu}

    \begin{itemize}[<+->]

        \item Úvodní přednáška -- \textbf{teď}

        \begin{itemize}[<+->]

            \item Technický úvod, do je/není AI

            \item Základy strojového učení a neuronových sítí

            \item Ukázky z~počítačového vidění a zpracování jazyka

        \end{itemize}

        \item Přednášky: host mluví na dané téma \\
            \quad 24.2. \quad 10.3. \quad 24.3. \quad 14.4. \quad 28.4


        \item Semináře: téma předchozí přednášky, obvykle založené na čtení
            nějaké literatury \\
            \quad 3.3. \quad 17.3. \quad 31.3. \quad 21.4. \quad 5.5.

        \item Na závěr (12.5.) shrnutí a závěrečný test

    \end{itemize}

    \vspace{10pt}

    \centering
    \visible<8->{\fbox{\begin{minipage}{.9\textwidth}
        \Large Zápočet: \\ $\bullet$ \textbf{účast} na 4 přednáškách a 4 seminářích (pokud
            pandemie dovolí) \\ $\bullet$ \textbf{závěrečný test} (pouze informativní)
    \end{minipage}}}

\end{frame}

% ----------------------------------------------------------------------------

\begin{frame}{Program}

    \small
    \begin{tabular}{cl}
        \toprule
    Datum   &   Program \\ \midrule
    17. 2.  & Úvodní přednáška \\ \midrule
    24. 2.  & Dita Malečková (AI v kontextu umění) \\
    3. 3.   & Seminář 1 \\ \midrule
    10. 3.  & David Černý (Ústav státu a práva, Ústav informatiky AV ČR, AI v kontextu etiky) \\
    17. 3.  & Seminář 2 \\ \midrule
    24. 3.  & Veronika Macková (FSV UK, AI v kontextu žurnalistiky) \\
    31. 3.  & Seminář 3 \\ \midrule
    14. 4.  & Petr Kajzar (1. LF UK, AI v kontextu medicíny) \\
    21. 4.  & Seminář 4 \\ \midrule
    28. 4.  & František Štěch (ETF UK, AI v kontextu náboženství) \\
    5. 5.   & Seminář 5 \\ \midrule
    12. 5.  & Závěrečné shrnutí a test \\
    \bottomrule
    \end{tabular}


\end{frame}

