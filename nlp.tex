% ----------------------------------------------------------------------------
\section[Zpracování jazyka]{Zpracování přirozeného jazyka}
% ----------------------------------------------------------------------------

\begin{frame}{Co je zpracování přirozeného jazyka}

    \begin{center}

        \Large Úlohy, pro jejichž řešení je potřeba (do nějaké míry) \\ \textbf{rozumět
        lidskému jazyku}

    \end{center}

    \begin{itemize}[<+->]

        \item<2-> základní řešení lze obvykle jednoduše naprogramovat

        \item<3-> pro zvýšení úspěšnosti jsou potřeba stále složitější pravidla

        \item<4-> od určité úrovně si neporadíme bez strojového učení

    \end{itemize}

\end{frame}

% ----------------------------------------------------------------------------

\begin{frame}{Vyhledávání odpovědí na otázky}

    \centering
    \includegraphics[scale=.26]{img/qa.png} \\
    {\tiny Screenhot z~dema Allen Institute for AI \url{https://demo.allennlp.org/reading-comprehension/transformer-qa}}

\end{frame}

% ----------------------------------------------------------------------------

\begin{frame}{Kontrola pravopisu}

    \centering
    \includegraphics[scale=.3]{img/grammarly.png} \\
    {\tiny Zdroj: Webová reklama grammarly.com}

\end{frame}

% ----------------------------------------------------------------------------

\begin{frame}{Strojový překlad}

    \centering
    \includegraphics[scale=.4]{img/lindat.png} \\
    {\tiny Zdroj: Překladač CUBBIT \url{https://lindat.mff.cuni.cz/services/translation} \citep{popel2020transforming}}

\end{frame}

% ----------------------------------------------------------------------------

\begin{frame}{Entity extraction}

    \includegraphics[scale=.4]{./img/mona_lisa.png} \\
    {\tiny Zdroj: \url{https://dandelion.eu/semantic-text/entity-extraction-demo}}

    \vspace{10pt}

    \visible<2->{Podúlohy:}
    \begin{enumerate}

        \item<3-> Named entity recognition: nalézt v~textu řetězce, co obsahují
            pojmenované entity

        \item<4-> Entity linking: co entity znamenají (první pád, odkaz do
            databáze/na Wikipedii)

    \end{enumerate}

\end{frame}

% ----------------------------------------------------------------------------

\begin{frame}{Koncept embeddingu}

    \begin{columns}

    \column{.45\textwidth}
    \begin{itemize}[<+->]

        \item neuronové sítě potřebují spojité vstupy

        \item one-hot vektor: očíslujeme slova, 0/1 indikuje slovo \\[1em]
            \scalebox{.9}{\input{img/onehot.tex}}

        \item první vrstva = násobení maticí

    \end{itemize}
    \column{.45\textwidth}

        \visible<4->{
        \centering
        \scalebox{4}{\rotatebox{-90}{\Huge $\Rsh$}}

        \vspace{5pt}

        {\Large
        Každé \textbf{slovo} (nebo jiná jednotka) je reprezentovaná mnohodimenzionálním
        \textbf{vektorem}}}

        \vspace{15pt}

        \begin{itemize}

            \item<5-> učí se ,,mimochodem`` z dat

            \item<6-> mají zajímavé vlastnosti

        \end{itemize}

    \end{columns}

\end{frame}

% ----------------------------------------------------------------------------

\begin{frame}{Ukázka z překladače}
    \hspace*{-10pt}\includegraphics{./plots/tsne.pdf} \\
    \tiny Zdroj: \citet{rozhledy}
\end{frame}

% ----------------------------------------------------------------------------

\begin{frame}{Princip transformeru}

    \begin{columns}
    \column{.45\textwidth}

        \scalebox{.8}{\begin{tikzpicture}

    \foreach \i in {1,...,8}{
			\node[draw,inner sep=7pt,fill=Magenta!80] (A\i) at (\i, 0) {\i};
			\node[draw,inner sep=7pt,fill=ForestGreen!80] (B\i) at (\i, 2.25) {\i};
			\node[draw,inner sep=7pt,fill=Cerulean!80] (C\i) at (\i, 4.5) {\i};
	}

    \foreach \i in {1,...,8}{
		\foreach \j in {1,...,8}{
		    \pgfmathsetmacro{\lnW}{0.1 + random(0,10)/10};
			\draw[->, line width=\lnW, color=Gray, -{latex[width=2mm,length=3mm]}] (A\i.north) -- (B\j.south);
		    \pgfmathsetmacro{\lnW}{0.1 + random(0,10)/10};
			\draw[->, line width=\lnW, color=Gray, -{latex[width=2mm,length=3mm]}] (B\i.north) -- (C\j.south);
		}
	}

\end{tikzpicture}
}

    \column{.45\textwidth}
        \begin{itemize}[<+->]

            \item Mezi klasickými vrstvami tzv. \textbf{self-attention}

            \item Každé slovo se ,,podívá'' na ostatní slova a vezme si z~něj
                relevantní informace

            \item Na začátku: vektor reprezentuje \textbf{izolované} slovo \\
                Na konci: vektor reprezentuje slovo \textbf{v~kontextu} věty

        \end{itemize}

    \end{columns}

    \vspace{10pt}
    \visible<4->{\begin{center}Transformer původně představili \citet{vaswani2017attention} původně pro strojový překlad.\end{center}}

\end{frame}

% ----------------------------------------------------------------------------

%\begin{frame}{Encoder-decoder: Strojový překlad}
%
%    \begin{center}
%    \includegraphics[scale=0.55]{img/encoder_decoder.pdf}
%    \end{center}
%
%    \vspace{10pt}
%
%    \centering
%    \visible<2->{
%        Dekodér sbírá informace \textbf{z~předchozích slov} a \textbf{z~enkodéru}} \\
%    \tiny Koncept enkodéru-dekodéru: \citep{kalchbrenner2013recurrent,sutskever2014sequence,bahdanaou2015neural}
%
%%\end{frame}
%
% ----------------------------------------------------------------------------
%\begin{frame}{Zkreslení z dat}
%\end{frame}

% ----------------------------------------------------------------------------

\begin{frame}{Předtrénované modely v NLP}

    \centering
    \begin{tikzpicture}

            \visible<3->{\draw[dashed,Gray] (0, -100pt) -- (0, 100pt);}

            \node[draw,fill=Orange!80,inner sep=15pt] (bert) {\begin{tabular}{c}Language \\ model\end{tabular}};

            \visible<2->{
            \node (pre) [left=50pt of bert.west] {\large\bf 1. Pre-training};
            \node (pre2) [below=0pt of pre.south] {\small (jednou, na velkých datech, trvá dlouho)};
            \node[draw,ellipse,inner sep=1pt,fill=Gray!20] (unlabeled) [above left=30pt and 30pt of bert.north west] {\begin{tabular}{c}prostý \\ text\end{tabular}};
            \draw[->] (unlabeled.east) to[out=0,in=90] (bert.north);
            \node[draw,ellipse,inner sep=3pt,fill=Gray!20] (mlm) [below left=30pt and 30pt of bert.south west] {\begin{tabular}{c}masked LM  \\ \hline causal LM\end{tabular}};
            \draw[->] (bert.south) to[out=270,in=0] (mlm.east);}

            \visible<3->{
            \node (fine) [right=50pt of bert.east] {\large\bf 2. Finetuning};
            \node (fine2) [below=0pt of fine.south] {\small (malá data, rychle)};
            \node[draw,ellipse,inner sep=1pt,fill=Dandelion] (labeled) [above right=30pt and 30pt of bert.north east] {\begin{tabular}{c}data pro \\ specifickou \\ úlohu\end{tabular}};
            \draw[->] (labeled.west) to[out=180,in=90] (bert.north);
            \node[draw,ellipse,inner sep=3pt,fill=Dandelion] (loss) [below right=30pt and 30pt of bert.south east] {\begin{tabular}{c}task-specific \\ objective\end{tabular}};
                    \draw[->] (bert.south) to[out=270,in=180] (loss.west);}


    \end{tikzpicture}
\end{frame}

%\begin{frame}{Předtrénování: Masked Language Model}
%
%    \centering
%    \fbox{\large
%    \colorbox{lightgray}{All}
%    \colorbox{lightgray}{human}
%    \colorbox{lightgray}{being}
%    \colorbox{lightgray}{are}
%    \colorbox{lightgray}{born}
%    \only<1>{\colorbox{lightgray}{free}}%
%    \only<2>{\colorbox{GreenYellow}{free}}%
%    \only<3>{\colorbox{Rhodamine!30}{\texttt{MASK}}}%
%    \only<4>{\colorbox{Dandelion}{hairy}}%
%    \only<5->{\colorbox{GreenYellow}{free}}
%    \colorbox{lightgray}{and}
%    \colorbox{lightgray}{equal}
%    \colorbox{lightgray}{in}
%    \colorbox{lightgray}{dignity}
%    \colorbox{lightgray}{and}
%    \colorbox{lightgray}{rights}}
%
%    \vspace{20pt}
%
%    \begin{enumerate}
%
%        \item<2-> Náhodně vybereme slovo $\rightarrow$
%            \colorbox{GreenYellow}{free}
%
%        \item<3-> S~pravděpodobností 80\% ho nahradíme značku
%            \colorbox{Rhodamine!30}{\texttt{MASK}}
%
%        \item<4-> S~pravděpodobností 10\% ho nahradíme náhodným slovem $\rightarrow$
%            \colorbox{Dandelion}{hairy}
%
%        \item<5-> S~pravděpodobností 10\% ho necháme, jak je $\rightarrow$
%            \colorbox{GreenYellow}{free}
%
%    \end{enumerate}
%
%		\vspace{10pt}
%
%    \visible<6->{
%    Model se snaží uhádnout chybějící slovo \colorbox{GreenYellow}{free}}
%    \visible<7->{
%    \ldots aby to dokázal musí nějak ,,rozumět'' zbytku věty.}
%
%    \vspace{20pt}
%    \tiny Myšlenka Masked Language model, viz. \citet{devlin2019bert}
%
%\end{frame}

% ----------------------------------------------------------------------------

\begin{frame}{Pre-training -- Finetuning}

    \centering

    {\huge
		Většina state-of-the-art řešení v NLP používá \textbf{předtrénovaný} model
		občas dotrénovaný na datech specifických pro úlohu.}

    \vspace{30pt}

	Trend vývoje: lepší předtrénované modely, nižší potřeba specifických dat,
		instruction tuning

\end{frame}