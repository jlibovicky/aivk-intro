\section{Počítačové vidění}

\begin{frame}{Úlohy počítačového vidění}

    \vspace{-5pt}
    \begin{center}
        \large Cokoli, co se dá zjistit z obrazu, např. \ldots
    \end{center}

    \visible<2->{
    \begin{columns}[t]
        \column{.3\textwidth}
        \centering \textbf{Rozpoznávání objektů} \\[1ex]
        \includegraphics[width=.98\textwidth]{img/object_detection.png} \\
        {\tiny Zdroj: \url{https://pjreddie.com/darknet/yolo}}

        \column{.3\textwidth}
        \centering \textbf{Odhad hloubky v obrazu} \\[1ex]
        \includegraphics[width=.98\textwidth]{img/monocular_depth.png} \\
        {\tiny Zdroj: \citet[s.\ 9250, obr.\ 1]{pillai2019super}}

        \column{.3\textwidth}
        \centering \textbf{Poloha lidského těla} \\[1ex]
        \includegraphics[width=.98\textwidth]{img/pose_detection.png} \\

        \begin{minipage}{\textwidth}
        \tiny Zdroj: \href{https://towardsdatascience.com/realtime-multiple-person-2d-pose-estimation-using-tensorflow2-x-93e4c156d45f}{\tt https://towardsdatascience.com/\\ realtime-multiple-person-2d-pose- \\ estimation-using-tensorflow2- \\ x-93e4c156d45f}
        \end{minipage}

    \end{columns}}

    \visible<3->{\begin{center}
        \large \ldots pro člověka většinou triviální úlohy
    \end{center}}

\end{frame}

% ----------------------------------------------------------------------------

\begin{frame}{Úlohy počítačového vidění}

    \visible<1->{\begin{center}
        \large \ldots ale i vyžadující expertní znalost.
    \end{center}}

    \visible<2->{
    \begin{columns}[t]
        \column{.45\textwidth}
        \centering \textbf{Detekce tuberkulózy} \\[1ex]
        \includegraphics[scale=.13]{img/tuberculosis.png} \\
        {\tiny Zdroj: \citet[obr.\ 3]{Pasa-et-al-2019}}

        \column{.45\textwidth}
        \centering \textbf{Katalogizace galaxií} \\[1ex]
    \includegraphics[scale=.15]{img/galaxies.png} \\
        {\tiny Zdroj: \citep[obr.\ 4]{gonzales2018galaxy}}
    \end{columns}}

    \centering
    \visible<3->{Pro počítač žádný rozdíl.}

\end{frame}

% ----------------------------------------------------------------------------

\begin{frame}{Konvoluční sítě}

    \begin{minipage}{.45\textwidth}
        \begin{itemize}[<+->]

            \item Základní nástroj pro zpracování obrazu neuronovými sítěmi

            \item Okénko, které jede přes obrázek a detekuje nějaké vzorce

            \item Mnoho vrstvev $\rightarrow$ vzorce vzorců vzorců\ldots

            \item Model si vytváří potřebné abstrakce

        \end{itemize}
    \end{minipage}\begin{minipage}{.45\textwidth}
        \centering
        \scalebox{0.7}{\input{img/convolution2d.pdf_tex}}
    \end{minipage}

\end{frame}

% ----------------------------------------------------------------------------

\begin{frame}{Hluboké konvoluční sítě}

    \begin{center}
        \scalebox{0.4}{\input{./img/alexnet.pdf_tex}}
    \end{center}

    \begin{itemize}[<+->]

        \item Hluboká architektura: konvoluce, max-pooling, residual connections,
            batch normalization, 50--150 vrstev

        \item Typicky ,,předtrénováno'' na velkých datech, pro jednotlivé úlohy
            se ,,dotrénovávají'' poslední vrstvy

    \end{itemize}

\end{frame}

% ----------------------------------------------------------------------------

\begin{frame}{ImageNet: rozpoznávání objektů}

    \begin{center}
        Dataset, který pomohl odstartovat revoluci ve strojovém učení.
    \end{center}

    \centering

    \includegraphics[scale=.34]{./img/imagenet.png} \\
    {\tiny Zdroj: \citet[obr.\ 1]{deng2009imagenet}}

    \visible<2->{14M obrázků, 1000 různých tříd}

\end{frame}

% ----------------------------------------------------------------------------

\begin{frame}{ImageNet challenge}

    \centering
    \begin{tikzpicture}[gnuplot]
%% generated with GNUPLOT 5.2p8 (Lua 5.3; terminal rev. Nov 2018, script rev. 108)
%% Po 31. ledna 2022, 18:45:58
\path (0.000,0.000) rectangle (13.000,7.000);
\gpcolor{color=gp lt color border}
\gpsetlinetype{gp lt border}
\gpsetdashtype{gp dt solid}
\gpsetlinewidth{1.00}
\draw[gp path] (1.504,0.616)--(1.531,0.616);
\draw[gp path] (12.447,0.616)--(12.420,0.616);
\node[gp node right] at (1.320,0.616) {0.00};
\draw[gp path] (1.504,1.629)--(1.531,1.629);
\draw[gp path] (12.447,1.629)--(12.420,1.629);
\node[gp node right] at (1.320,1.629) {0.05};
\draw[gp path] (1.504,2.641)--(1.531,2.641);
\draw[gp path] (12.447,2.641)--(12.420,2.641);
\node[gp node right] at (1.320,2.641) {0.10};
\draw[gp path] (1.504,3.654)--(1.531,3.654);
\draw[gp path] (12.447,3.654)--(12.420,3.654);
\node[gp node right] at (1.320,3.654) {0.15};
\draw[gp path] (1.504,4.666)--(1.531,4.666);
\draw[gp path] (12.447,4.666)--(12.420,4.666);
\node[gp node right] at (1.320,4.666) {0.20};
\draw[gp path] (1.504,5.679)--(1.531,5.679);
\draw[gp path] (12.447,5.679)--(12.420,5.679);
\node[gp node right] at (1.320,5.679) {0.25};
\draw[gp path] (1.504,6.691)--(1.531,6.691);
\draw[gp path] (12.447,6.691)--(12.420,6.691);
\node[gp node right] at (1.320,6.691) {0.30};
\node[gp node center] at (2.572,0.308) {$2011$};
\node[gp node center] at (3.906,0.308) {$2012$};
\node[gp node center] at (5.241,0.308) {$2013$};
\node[gp node center] at (6.575,0.308) {$2014$};
\node[gp node center] at (7.910,0.308) {$2015$};
\node[gp node center] at (9.244,0.308) {$2016$};
\node[gp node center] at (10.579,0.308) {$2017$};
\draw[gp path] (1.504,6.691)--(1.504,0.616)--(12.447,0.616)--(12.447,6.691)--cycle;
\node[gp node center,rotate=-270] at (0.016,3.653) {5-best error rate};
\gpfill{rgb color={0.114,0.188,0.459}} (2.071,0.616)--(3.073,0.616)--(3.073,5.821)--(2.071,5.821)--cycle;
\gpcolor{rgb color={0.114,0.188,0.459}}
\draw[gp path] (2.071,0.616)--(2.071,5.820)--(3.072,5.820)--(3.072,0.616)--cycle;
\gpfill{rgb color={0.114,0.188,0.459}} (3.406,0.616)--(4.408,0.616)--(4.408,3.718)--(3.406,3.718)--cycle;
\draw[gp path] (3.406,0.616)--(3.406,3.717)--(4.407,3.717)--(4.407,0.616)--cycle;
\gpfill{rgb color={0.114,0.188,0.459}} (4.740,0.616)--(5.742,0.616)--(5.742,2.953)--(4.740,2.953)--cycle;
\draw[gp path] (4.740,0.616)--(4.740,2.952)--(5.741,2.952)--(5.741,0.616)--cycle;
\gpfill{rgb color={0.114,0.188,0.459}} (6.075,0.616)--(7.077,0.616)--(7.077,2.117)--(6.075,2.117)--cycle;
\draw[gp path] (6.075,0.616)--(6.075,2.116)--(7.076,2.116)--(7.076,0.616)--cycle;
\gpfill{rgb color={0.114,0.188,0.459}} (7.409,0.616)--(8.411,0.616)--(8.411,1.339)--(7.409,1.339)--cycle;
\draw[gp path] (7.409,0.616)--(7.409,1.338)--(8.410,1.338)--(8.410,0.616)--cycle;
\gpfill{rgb color={0.114,0.188,0.459}} (8.744,0.616)--(9.746,0.616)--(9.746,1.223)--(8.744,1.223)--cycle;
\draw[gp path] (8.744,0.616)--(8.744,1.222)--(9.745,1.222)--(9.745,0.616)--cycle;
\gpfill{rgb color={0.114,0.188,0.459}} (10.078,0.616)--(11.080,0.616)--(11.080,1.073)--(10.078,1.073)--cycle;
\draw[gp path] (10.078,0.616)--(10.078,1.072)--(11.079,1.072)--(11.079,0.616)--cycle;
\gpcolor{color=gp lt color border}
\node[gp node left] at (1.836,6.128) {\footnotesize~err=.257};
\node[gp node left] at (3.170,4.025) {\footnotesize~err=.153};
\node[gp node left] at (4.505,3.260) {\footnotesize~err=.115};
\node[gp node left] at (5.839,2.424) {\footnotesize~err=.074};
\node[gp node left] at (7.174,1.646) {\footnotesize~err=.035};
\node[gp node left] at (8.508,1.530) {\footnotesize~err=.029};
\node[gp node left] at (9.843,1.380) {\footnotesize~err=.022};
\node[gp node left] at (1.836,6.497) {\footnotesize~\citet{sanchez_high-dimensional_2011}};
\node[gp node left] at (3.170,4.394) {\footnotesize~\citet{krizhevsky_imagenet_2012}};
\node[gp node left] at (4.505,3.629) {\footnotesize};
\node[gp node left] at (5.839,2.793) {\footnotesize~\citet{simonyan_very_2014}};
\node[gp node left] at (7.174,2.015) {\footnotesize~\citet{he_deep_2016}};
\node[gp node left] at (8.508,1.899) {\footnotesize};
\node[gp node left] at (9.843,1.749) {\footnotesize~\citet{hu_squeeze-and-excitation_2017}};
\node[gp node left] at (3.170,4.764) {\footnotesize\it~AlexNet};
\node[gp node left] at (5.839,3.163) {\footnotesize\it~VGG19};
\node[gp node left] at (7.174,2.385) {\footnotesize\it~ResNet};
\node[gp node left] at (9.843,2.119) {\scriptsize\it~Squeeze~and~Excitation};
\draw[gp path] (1.504,6.691)--(1.504,0.616)--(12.447,0.616)--(12.447,6.691)--cycle;
%% coordinates of the plot area
\gpdefrectangularnode{gp plot 1}{\pgfpoint{1.504cm}{0.616cm}}{\pgfpoint{12.447cm}{6.691cm}}
\end{tikzpicture}
%% gnuplot variables


\end{frame}

% ----------------------------------------------------------------------------

\begin{frame}{Předtrénované reprezentace}

    \begin{columns}

        \column{.45\textwidth}
        \centering

        Nejpodobnější vektory k reprezentaci z poslední vrstvy AlexNetu.

        \includegraphics[scale=0.25]{./img/alexnet_neighbors.png} \\
        {\tiny Zdroj: \citet[obr.\ 4]{krizhevsky_imagenet_2012}}

        % TODO: vyznačit sloupec a neighbors
        % TODO: zopakovat obrázek AlexNetu 


        \column{.45\textwidth}
    \begin{itemize}[<+->]

        \item Klasifikace obrázků se učí obecné rysy

        \item Prakticky všechny úlohy začínají s předtrénovaným modelem

        \item Postupně se nahrazuje unsupervised metodami

    \end{itemize}


    \end{columns}

\end{frame}

% ----------------------------------------------------------------------------

\begin{frame}{Problémy ImageNetu}

    \begin{itemize}[<+->]

        \item Založeno na lexikální databázi z 90.\ let -- reflektuje o čem se
            mluvilo v USA před 40 lety

        \item Není kulturně neutrální: anglické slovo \emph{shovel} v~severní
            Americe a v~jižní Africe \\ {\tiny obrázek z prezentace
            \citet{liu-etal-2021-visually}} \\
            \includegraphics[scale=.35]{img/shovel.png}

        \item Obsahuje obrázky ke nutně stereotypním konceptům jako {\tiny \citep[str.\ 109]{crawford2021atlas}} \\
            \quad\it ,,alcoholic,'' ,,ape-man,'' ,,crazy,'' ,,hooker''
    \end{itemize}

\end{frame}

% ----------------------------------------------------------------------------

%\begin{frame}{Rozpoznávání obličejů}
%\end{frame}


% ----------------------------------------------------------------------------